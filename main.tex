\documentclass[a4paper]{memoir}

\usepackage[utf8]{inputenc}
\usepackage{geometry}
\usepackage{amsmath, amsfonts, amssymb, amsthm, mathtools}
\usepackage{graphicx}
\usepackage{parskip}
\usepackage{fancyhdr}
\usepackage{lastpage}
\usepackage{optidef}
\usepackage{hyperref}
\usepackage{tikz}

\usepackage[
style=authoryear]
{biblatex}
\addbibresource{sources.bib}


\chapterstyle{ger}
\maxsecnumdepth{subsection}
\maxtocdepth{subsection}

\pagestyle{fancy}
\fancyhf{}
\lhead{\rightmark}
\rhead{\leftmark}
\rfoot{Page \thepage \hspace{1pt} of \pageref{LastPage}}

\renewcommand{\headrulewidth}{1pt}
\renewcommand{\footrulewidth}{1pt}

\title{Specialization Project}
\author{Ulrik Bernhardt Danielsen}

\theoremstyle{plain}
\newtheorem{theorem}{Theorem}[section]
\newtheorem{lemma}{Lemma}[section]
\newtheorem{corollary}{Corollary}[theorem]
\newtheorem{proposition}{Proposition}[section]

\theoremstyle{definition}
\newtheorem{definition}{Definition}[section]
\newtheorem{example}{Example}[section]

\theoremstyle{remark}
\newtheorem*{remark}{Remark}

\begin{document}

\maketitle

\tableofcontents*
\clearpage


\section{Steps:}
\begin{enumerate}
        \item z-scoring (standardization \textcolor{red}{although dividing by standard deviation is commented out})
        \item Smoothing by 3d splines
        \item Detrending around the smoothed curve \textcolor{red}{(Why do we need to do this? Do we expect trends?)}
        \item Morlet continuous wavelet transformation
                \begin{enumerate}
                        \item Hypothesis testing against 1st order autoregressive process
                        \item Smoothing of power across scales
                        \item Scale-averaged wavelet power
                        \item Rescaling features
                        \item Concatenating trend data and power spectrum
                \end{enumerate}
        \item \underline{Feature vectors were downsampled in time at 1 Hz and pooled acreoss animals and conditions}. \textcolor{red}{Where is this done in the code?}
        it
        \item PCA, reducing to feature dimension explaining at least 95\% of the variance
        \item t-SNE
        \item Watershed segmentation
\end{enumerate}
\newpage



\chapter{Introduction}
\section{Motivation}
The human brain is an incredibly complex structures that researchers have been trying to understand for a long time.
One way to gain information about how the brain operates is to study its neurons.
Neurons are cells which can communicate with each other through synapses.
This communication are electric signals and can be recorded.
\textcolor{red}{Source?}
At Kavli Institute for Systems Neuroscience at NTNU they are interested in relating these neural spike recordings to the behavior in rats.
This in turn begs the question of how rats behave.
Manually labelling video recordings of rats running around seems a tedious and unfruitful endeavor.
Additionally it introduces bias in our prior assumptions of how the rats behave, and which activities they engage in.
Thus, a methodology for automatically detecting distinct behaviours is needed.

\section{Previous work}
\textcolor{red}{Is this necessary?}



\chapter{Theory}
\section{Time series analysis}
We define time series as a realization $y_t = \{ y_{t_1}, y_{t_2}, \hdots, y_{t_n} \}$ of a stochastic process $Y(w, t)$, where $w \in \Omega$, $\Omega$ being the sample space,  and $t \in \mathbb{Z}$, $\mathbb{Z}$ being the chosen index set  \cite{wei}.
It is an ordered series of random variables which can be described completely by its joint probability function
\begin{equation*}
        F_{t_1,\hdots, t_n}(x_1, \hdots, x_n) = \text{Pr}\{ y_{t_1} \leq x_{1}, \hdots, y_{t_n} \leq x_n) \}.
\end{equation*}
The mean and variance function of a time series $y$ are defined as
\begin{equation}\label{eq:mean_func}
        \mu_t = E(y_t)  
\end{equation}
and
\begin{equation*}
        \sigma_t^2 = E(y_t - \mu_t)^2.
\end{equation*}

Given two random variables in the series $y_{t_1}$ and $y_{t_2}$, we define the covariance function and correlation function as
\begin{equation}\label{eq:acv}
        \gamma (t_{1}, t_2) = E[(y_{t_1} - \mu_{t_1})(y_{t_2} - \mu_{t_2})]
\end{equation}
and 
\begin{equation}\label{eq:acf}
        \rho(t_1, t_2) = \frac{\gamma ( t_1, t_2)}{\sqrt{\sigma_{t_1}^2}\sqrt{\sigma_{t_2}^{2}}}.
\end{equation}

\subsection{Stationarity}
A time series $y_t$ is $n$th-order stationary if for any shift $h$ and indexes $t_1, t_2, \hdots, t_n$ if 
\begin{equation}\label{eq:nth_stationary}
        F_{y_{t_1}, \hdots, y_{t_n}}(x_1, \hdots, x_n) = F_{y_{t_{1}+h}, \hdots, y_{t_{n} +h}} (x_1, \hdots, x_n).
\end{equation}
If \eqref{eq:nth_stationary} holds for all $n$, the time series is called \textit{strictly} stationary.
We also define a $n$th-order \textit{weakly} stationary time series $y_t$ if the first $n$ joint moments are finite and time invariant.
Specifically we define the second-order weakly stationary, i.e. with constant and time invariant mean function \eqref{eq:mean_func}, and where the covariance function \eqref{eq:acv} is solely a function of the time difference, as \textit{covariance} stationary.
When the covariance function between $t_1, t_2$ can be written as a function of the time difference $h = |t_1 - t_2|$, i.e. $\gamma (t_1, t_2) = \gamma (h) = \gamma_h$, we call it an \textit{autocovariance} function. 
The same is true for the correlation function \eqref{eq:acf}, which when is a function of the time difference is called an \textit{autocorrelation} function (ACF).
Figure \ref{fig:time_series_example} shows an example of a time series.
As the mean seem to increase with $t$ it is non-stationary.

\begin{figure}[tb]
        \centering
        \includegraphics[width=\linewidth]{./code/figures/time_series_example.pdf}
        \caption{Example of a non-stationary time series.}
        \label{fig:time_series_example}
\end{figure}


\subsection{Detrending}
Many methods for analysing and processing time series requires stationarity \cite{shumway}.
If the series is non-stationary, we can split the it into one stationary and one non-stationary part called the \textit{trend}.
Mathematically we write it as 
\begin{equation*}
        y_t = \mu_t + x_t,
\end{equation*}
where $x_t$ denotes the stationary part and $\mu_t$ the trend.
The process of finding $\mu_t$ and then computing $x_t = y_t - \mu_t$ is called \textit{detrending}.
Detecting the trend can be done in many ways, for instance using regression techniques or smoothing.
The simplest way is to assume a linear trend, $\mu_t = \beta_0 + \beta_1 t$ and estimate the parameters using least squares.
In figure \ref{fig:time_series_example_with_trend} the linear regression fit is shown, showing an upwards in the time series.


\begin{figure}[tb]
        \centering
        \includegraphics[width=\linewidth]{./code/figures/time_series_example_with_trend.pdf}
        \caption{}
        \label{fig:time_series_example_with_trend}
\end{figure}




\subsection{Fourier analysis}
\subsection{Spectral analysis}
\subsection{Wavelet transformation}
Morlet wavelet—a sine wave that is "windowed" (i.e., multiplied point by point) by a Gaussian

\section{Piecewise polynomials}
Suppose we have an interval $[a,b]$ divided into $M$ contiguous subintervals.
The connecting edges of the subintervals $a = \xi_0, \xi_1, \hdots, \xi_{M - 1}, \xi_{M} = b$ are called knots.
On each of the intervals $[\xi_i, \xi_{i+1}], i = 0, \hdots, M-1$ we define a polynomial $p_i (t)$.
The function
\begin{equation*}
        f(t) = 
                \begin{cases}
                        p_0(t), &  t \in [\xi_0, \xi_{1}) \\
                        p_1(t), & t \in [\xi_1, \xi_2)  \\
                        & \vdots \\
                        p_{M-1}(t), & t \in [\xi_{M-1}, \xi_{M}]  \\
                \end{cases}
\end{equation*}
is called a \textit{piecewise polynomial}.


\subsection{Splines}
In the definition of piecewise polynomials no restrictions are made on the polynomials, they are allowed to take any form.
As in \cite{quarteroni} we define a \textit{spline} $s_k(t)$ of order $k$ on the interval $[a,b]$ as a piecewise polynomial where
\begin{align*}
        &s_k(t) \in \mathcal{P}^k , \quad t \in [\xi_i, \xi_{i+1}],\quad i = 0, 1, \hdots, M-1 \\
        &s_k(t) \in \mathcal{C}^{k - 1}[a, b].
\end{align*}
I.e., the spline consists of piecewise polynomials of order $k$ and has continuous derivatives up to order $k - 1$.
A common choice is letting $k = 3$, providing continuous second derivatives over the interval.
This is called \textit{cubic} splines, and are often considered sufficiently smooth for function approximations.
It is also common to add curvature constraints at the endpoints, $s_3''(a) = s_3''(b)$, arriving at the \textit{natural} cubic splines.

\subsection{Regression splines}
Suppose now we have data points $y_{t_1}, y_{t_2}, \hdots, y_{t_n}$ on $[a = t_1, b = t_n]$. 
A spline of order $k$ with chosen knots at $a = t_1 = \xi_0, \xi_1, \hdots, \xi_{M} = t_n = b$ can be parameterized as 
\begin{equation*}
        s_k(t) = \sum_{i = 1}^{M + K} \beta_i h_i(t),
\end{equation*}
where the functions $h_i$ are the truncated-power basis set
\begin{align*}
        h_j(t) &= t^{j - 1}, \ j = 1, \hdots, k+1, \\
        h_{k+1+l}(t) &= (t - \xi_l)_+^k, \ l = 1, \hdots, M-1,
\end{align*}
with $(t)_+ = \max_{} \{ t, 0 \}$ \cite{hastie}.
The parameters $\beta_i$ can be found using least squares.
An example of cubic spline regression are shown in figure \ref{fig:cubic_splines}.

\begin{figure}[tb ]
        \centering
        \includegraphics[width=\linewidth]{./code/figures/cubic_splines.pdf}
        \caption{Two cubic splines fitted using least square regression.
        Observe that the red spline with 50 knots (including endpoints) fits the data closer than the green spline with 10 knots.}
        \label{fig:cubic_splines}
\end{figure}




\section{Dimensionality reduction}
\subsection{Principal Component Analysis}
\subsection{t-Stochastic Neighbor Embedding}

\section{Watershed segmentation}

\chapter{Methodology}
What is done in practice.
Discussion  of choices made.
\section{Input data}
The starting point of the analysis is a collection of $n$ separate time series of equal length $m$ \textcolor{red}{right?}.
We denote these $\textbf{Y} = \{ y_1(t), \hdots, y_n(t) \}$, where $t \in \{ t_1, t_2, \hdots, t_m \}$.
\section{Feature extraction}
\section{Manifold embedding}








\newpage
\printbibliography
\end{document}



